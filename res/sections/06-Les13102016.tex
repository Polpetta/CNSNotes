% chapter 1 one the book
\section{Computer Security introduction}

Computer security is not simple, it involves a lot of other topics and
knowledge.

During the years, there was evolution of the attacks, because during the years
there was also a improvement of security. Security requires regular monitoring
to be sure that everything it's ok, you have to remember that security isn't a
product.

Sometime security is also seen as a impediment to using a system.

What are the key component of a security system?
\begin{itemize}
  \item Owner: the one that own the data
  \item Risk: risk of a \textbf{vulnerability} in the system
  \item Vulnerabilities
  \item Contermeasures: can stop some \textbf{vulnerabilites} but it can add new
one.
  \item Threat agents: give raise to \textbf{threats} to use
\textbf{vulnerabilities}
\end{itemize}

\paragraph*{Countermeasures} To mitigate threats we can use come
countermeasures, and we can:
\begin{itemize}
  \item Prevent attacks (for example blocking some IPs). In some case you can't
prevent
  \item Detect: when you can't prevent attacks you can detect intrusion/when
there is an attack (for example for a DoS you can track the source of the IP
requests)
  \item Recover: you should take some countermeasures to stop vulnerabilities,
but you have to take care of the new vulnerabilities that you can introduce.
\end{itemize}

\subsection{Scoper of Computer Security}
In every system there are some processes (for example OS or programs) that use
some data. It's important to controll the access to the data from the
processes. Access to the computer facility must be controlled (maybe with a
access policy).

\paragraph*{Network Security Attacks}
\begin{itemize}
  \item Classify as passive or active
  \item Passive attacks are eavesdropping (the attacker doesn't change the data
of the packets)
  \begin{itemize}
    \item Release of message contents
    \item Traffic analysis
    \item Are hard to detect so aim to prevent
  \end{itemize}
  \item Active attacks modify/fake data (the attacker change the data of the
packets)
  \begin{itemize}
    \item Masquerade
    \item Replay
    \item Modification
    \item Denial of Service
    \item Hard to prevent so aim to detect
  \end{itemize}
\end{itemize}

\subsection{Security Functional Requirements}
Tecnical measures means that ??
Sometime you need also to manage the security. For example in a office you know
that you can't put your password in the desk!

\subsubsection{Standards}
There are different standards:
\begin{itemize}
  \item X.800, Security Architecture for OSI
  \item Systematic way of defining requirements for security and characterizing
approaches to satisfyng them
  \item Defines:
  \begin{itemize}
    \item Security attacks - compromise security
    \item Securiy mechanism - act to detect, prevent, recover from attack
    \item Security service - counter security attacks
  \end{itemize}
\end{itemize}

%chapter 2 of the book
\section{Criptology Introduction}
\subsection{Symmetric Encryption}

The same key is used for encryption and for decryption.
The problem is that they have to share the key in some way. This is a huge
limitation, because you can't talk to someone without met it before.
\subsubsection{Arracking Symmetric Encryption}
\begin{itemize}
  \item Cryptanalysis
  \begin{itemize}
    \item Rely on nature of the algorithm
    \item Plus some knowledget of plaintext characteristics
    \item Event some sample plaintext-ciphertext paris
    \item Exploits characteristics of algoithm to deduce specific plaintext or
key
  \end{itemize}
  \item Brute force attack. The problem is that the longer is the key the
longer time it takes.
\end{itemize}

\subsection{ASymmetric Encryption}
Asymmetric keys are useful to exchange a simmetric key. Using asymmetric
encryption it's time-expensive.
