\chapter{Contextual Security}

\begin{itemize}
  \item It uses supplemental information to improve security decision at the
time the decisions are made
  \item Machine learning plus contextual information = security and privacy
solution
  \item Benefits:
  \begin{itemize}
    \item Increased effectiveness of information security decisions
    \item Greater efficiency of IT security technology and human resources
    \item More flwxibility to apply security where and when it's needed most
  \end{itemize}
\end{itemize}

How to achieve context with existing security and information technology?
\begin{itemize}
  \item Availability
  \item Efficient integration in real-time
  \item Accessibility to security analysts
\end{itemize}

For example, in financial institutions contextual security is used for most of
the time.

\paragraph*{Type of contexts} You can have different type of Contexts:
\begin{itemize}
  \item Ambient environment (sensor modalities and physical)
  \item Identity (user, group and organization)
  \item Device (type, virtual machine and physical, IP address)
  \item Network (packets, connection type/port and protocols)
  \item Content (files, dataase, executables, e-mail and inputs)
\end{itemize}

A compresence detection system consists of a prover and verifier, where the
verifier confirms:
\begin{itemize}
  \item Prover is legitimate
  \item Verifier and prover are in close proximity
\end{itemize}
Real world instances for compresence detection systems includes:
\begin{itemize}
  \item Passive keyless entry (PKE) system
  \item Contactless smartcard based access control systems
\end{itemize}
Sensor modalities includes:
\begin{itemize}
  \item Audio, Wifi, GPS and Bluetooth
  \item Temperature, humidity, gas and altitude/pressure
\end{itemize}
Ab effective Contextual Copresence Detection approach should exhibit:
\begin{itemize}
  \item Low false negatives (i.e., rejecting a co-presence instance; a measure
of usability)
  \item Low false positives (i.e., accepting a non copresence instance; a
measure of security)
\end{itemize}
