\section{Security}
Important principles about security:
\begin{itemize}
  \item It's difficult to provide a definition of security. Security it's not "a
  product", it's rather a "process", which needs to be managed properly. It
  involve a lot of procedure, more than "products".
  \item Nothing can be 100\% secure.
  For example with credit card, there is a given amount of fraud. Refund people
  from a fraud it's costly for a bank, but spend money to make the system more
  secure can be more expensive than refund people.
  \item The security of a system is equivalent to the security of it's less
secure component\footnote{Also know as rule of the weakest link}.
  \item Security by obscurity never works.
  \item Cryptography is a powerful tool but this it's not enough!\footnote{"The
protection provided by encryption is based on the fact that most people would
rather eat liver than do mathematics" - Bill Neugent}
  \item Do not rely on users\footnote{"Given a choice between dancing pigs and
security, users will pick dancing pigs everytime" - Prof. Ed Felten (Princeton
University)}.
\end{itemize}
To give a definition of a security it's better to list the basic security
properties that a secure system needs to have:
\begin{itemize}
  \item Confidentiality: to prevent unauthorized disclosure of the information
  \item Integrity: to prevent unauthorized modification of the information
  \item Availability: to guarantee access to information
  \item Authentication: to prove the claimed identity can be Data or Entity
authentication
  \item Non repudiation: to prevent false denial of performed actions
  \item Authorization: "What Alice can do"
  \item Auditing: to securely record evidence of performed actions
  \item Attack-tolerance: ability to provide some degree of service after
failures or attacks
  \item Disaster recovery
  \item ??
\end{itemize}

There are different security mechanisms, that we will see during the course.
For example randomness, hash functions, hash chain, MIC, MAC, HMAC, Time
Stamping are widely used.

In security we need to know who is the adversary of the system. Also, it's
important to know what adversary can do. There are also different type of
attack that an adversary can implement:
\begin{itemize}
  \item Passive: the attacker can only read any information
  \begin{itemize}
    \item Tempest (signal intelligence)
    \item Packet Sniffing
  \end{itemize}
  \item Active: the attacker can read, modify, generate and destroy the system.
\end{itemize}

Today, a common attack it's with the big data, where an attacker can
profile users.
