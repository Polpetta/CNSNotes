\section{Password Vulnerabilities}
There are different type of attaks:
\begin{itemize}
  \item Offline dictionary attack: for example LinkedIn some time ago
  \item Specific account attack: when an account of a specific person is
targetted %targetted?
  \item Popular password attack
  \item Passeord guessing against single user
  \item Workstation hijacking: an attacker can copy data from a computer
  \item Exploiting user mistakes: an example copy the password in a piece of
paper
  \item Exploiting multimple password use: you should use different passwords
for different systems
  \item Electronic monitoring
\end{itemize}

There are some countermeasures:
\begin{itemize}
  \item Stop unauthorized access to password file
  \item Intrusion detection measures
  \item Account lockout mechanisms
  \item Policies against using common passwords but rather hard to guess
passwords
  \item Training \& enforcement of policies
  \item Automatic workstation logout
  \item Encrypted network links
\end{itemize}

Passowrd are not stored in a database. Passowrd are hashed in a database. For
security reasons, a \textit{salt} is used with a password to make more
difficult for an attacker to guess the password given an hash.


The UNIX implementation is composed by:
\begin{itemize}
  \item Original scheme:
  \begin{itemize}
    \item 8 character password form 56-bit key
    \item 12 bit salt used to modify DES encryption into a one-way hash function
    \item 0 value repeatedly encrypted 25 times
    \item output translated to 11 character sequence
  \end{itemize}
  \item Improved Implementation:
  \begin{itemize}
    \item Have other, stronger, hash/salt variants
    \item Many systems now use MD5:
    \begin{itemize}
      \item With 48 bit salt
      \item Password length is unlimited
      \item Is hashed with 1000 times inner loop
      \item Produces 128 bit hash
    \end{itemize}
    \item OpenBSD uses Blowfish block cipher based hash algorithm called
Bcrypt: it uses 128 bit salt to create 192 bit hash value
  \end{itemize}
\end{itemize}
This implementation it's not secure: with 50 million tests an attacker can
decrypt it in 80 min.

\paragraph*{Password Cracking}
There are different type of attacks:
\begin{itemize}
  \item Dictionary attacks: try each word then obvious variants in large
dictionaty against hash in password file
  \item Rainbow table attacks. Steps:
  \begin{itemize}
    \item Precompute tables of hash values for all salts
    \item A mammoth table of hash values
    \item E.g. 1.4GB table cracks 99.9\% of alphanumeric Windows passwords in
13.8 secs
    \item Not feasible if larger salt is used
  \end{itemize}
\end{itemize}

\paragraph*{Password Choices}
\begin{itemize}
  \item Users may pack short passwords
  \begin{itemize}
    \item E.G. 3\% were 3 chars or less, easily guessed
    \item System can reject choices that are too short
  \end{itemize}
  \item Users may pick guessable passwords
\end{itemize}
What can we do:
\begin{itemize}
  \item Can block offline guessing attacks by denyinh access to encrypted
passowrd
  \begin{itemize}
    \item Make available only to privileged users
    \item Often using a separate shadow passord file
  \end{itemize}
  \item Still ave vulnerabilities
  \begin{itemize}
    \item Exploit OS bug
    \item Accident with permissions making it readable
  \end{itemize}
\end{itemize}
The first idea so it's to use a better password. There are some techniques:
\begin{itemize}
  \item User education: teach the user how to choose a password
  \item Computer-generated passwords: but are difficult to remember
  \item Reactive password checking
  \item Proactive password chacking: check if the passowrd follow some defined
rules
  \begin{itemize}
    \item Rule enforcement plus user advice, for example:
    \begin{itemize}
      \item 8+ chars, upper/lower/numeric/punctation
      \item May not suffice
    \end{itemize}
    \item Password cracker: time and space issues
    \item Markov Model
    \begin{itemize}
      \item Generates guessable password
      \item Hence reject any password it might generate
    \end{itemize}
    \item Bloom Filter
    \begin{itemize}
      \item Use to build table based on dictionaty using hashes
      \item Check desired password against this table
      \item Very used in constrained devices
    \end{itemize}
  \end{itemize}
\end{itemize}

\paragraph*{Smart Card}
\begin{itemize}
  \item Credit-card like
  \item Has own processor, memory, I/O ports
  \begin{itemize}
    \item Wired or wireless access by reader
    \item May have crypto co-processor
    \item ROM, PROM, E-PROM
  \end{itemize}
\end{itemize}

\section{Access Control}

The prevention of unauthorized use of a resource, including the prevention of
use of a resource in an unauthoried way.
