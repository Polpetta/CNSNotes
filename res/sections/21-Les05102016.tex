\chapter{CNN - Content Centric Networks}
TCP is a complicate protocol, and originally the Internet was designed to allow
people to use remote services easily. But in the last 20 years there were a lot
of deep changes in the nature of Internet communication. In fact originally
Internet was designed like a private club, but nowadays this isn't true. People
use Internet and produce a lot of information (social networks, video, etc).
Some of the original things are still here and are still used, e.g. emails, but
some new aspects and problems have been arising: IoT, cloud, etc.

In addition, Internet was designed for a small and friendly world, not for an
hostile one, and the story of the Internet security isn't a success story. A
lot of security solutions came later and they were retrofitted and incremental:
SSH, HTTPS, etc.

There's the need of a new Internet architecture: FIA (Future Internet
Architecture). For almost 150 years, in fact, communication means "a way to
connecting devices", but now the information itself is more important: what
matters is the content, not the host it come from. The action of pushing
something on the web is more rare than the action of pulling that information:
this is why the information matters, more that where it come from. Content
distribution is common, but actually it's not performed very well: the main goal
of CCN is to create a scalable content distribution. In a classical
architecture, content is required by the devices and this isn't scalable.

\section{Architecture}
Every content has a name (like URIs) and there can be three roles:
\begin{itemize}
  \item \textbf{consumer};
  \item \textbf{producer};
  \item \textbf{router}.
\end{itemize}
The same device can play different roles in different contexts. A consumer is
\textbf{interested} by some contents; the interest is like a request and the
consumer asks for some contents. There is no source or destination addresses.
After it produces, the content can be everywhere, and this is why there is
no destination address. The interest is forwarded from one router to another
until it reaches the producer. If the content doesn't exist, the producer
discards the interest. Otherwise it sends the content. The router may have (and
it should too) have a cache, thus it can store the contents for future
interests.

A router has:
\begin{itemize}
  \item \textbf{PIT}: Pending Interest Table;
  \item \textbf{CS}: Content Store (cache);
  \item \textbf{FIB}: Forwarding Information Base.
\end{itemize}

A router can understand and recognize duplicate interests by checking the PIT,
so it doesn't propagate duplicates and processes them when the content arrives:
it's impossible to receive the same content more than once if the router uses
PIT and CS.

\paragraph*{Security}
The approach to the security is different in CCN: a "normal" router doesn't
look into packets, and data are authenticated because they are emitted from
the endpoint of a secure pipe. In CCN, contents can came from everywhere and it
makes the access control more difficult. The producer couldn't know how popular
the content is, and the consumer has to trust the content it receives.

The content has three fields:
\begin{itemize}
  \item \textbf{Name}, which has to match with the name of the content. There's
no limit to the length of the name;
  \item \textbf{Data};
  \item \textbf{Signature}, performed with a public/private key system.
\end{itemize}
The digital signature provides integrity and data authenticity; data must be
hashed before they are signed.

CCN doesn't specify how to encrypt secure contents: it's all about the
application, which can restrict access control, for example:
\begin{itemize}
  \item encrypting one with a symmetric key;
  \item distributing the symmetric key to authorized consumers.
\end{itemize}
CCN is PKI-agnostic, so the network layer has not a mandatory security (which
also happens for the classical architecture of Internet): security is
application specific. Routers can verify content signature to check that
it's not a fake; they don't verify interests, because it would be an
unnecessary effort.

Consumers haven't prefix. This means that until they produce a content they
can't be attacked. However, consumers are required to verify content
signatures: they are stupid if they don't!

Name is very revealing! That's an important problem for privacy (privacy is an
enemy for security, and vice-versa). Another problem is caching: it's a kind of
an history.

There are two types of DoS attacks:
\begin{itemize}
  \item \textbf{interest flooding}\footnote{generating random interests};
  \item \textbf{content poisoning}\footnote{introduction of fake contents}. It
can be \textbf{reactive} (the content is compromised at the level of the
router) or \textbf{proactive} (a producer generates some fake contents and a
consumer, or more than one, requests that content; the fake will be cached and
that's it!).
\end{itemize}
In conclusion, cache helps attackers, but it's necessary!

\section{Complete roles view}
\begin{itemize}
  \item \textbf{Producer}:
  \begin{itemize}
    \item \textit{announces} name prefix;
    \item \textit{names and signs} content packets;
    \item \textit{injects contents into the network} by answering interests.
  \end{itemize}
  \item \textbf{Consumer}:
  \begin{itemize}
    \item \textit{generates interests} referring to the content by name;
    \item \textit{receives contents, verifies signature} and decrepit if
necessary;
  \end{itemize}
  \item \textbf{Router}:
  \begin{itemize}
    \item \textit{routes interests} based on name prefixes;
    \item \textit{remembers} where interests come from;
    \item \textit{caches content} (optionally).
  \end{itemize}
\end{itemize}
