\section{Cache Privacy in CCN}

There is a trade-of between usability and privacy.
\begin{itemize}
  \item Router content caching is good for performance
  \begin{itemize}
    \item Better bandwidth utilization
    \item Lower latency, because you only need to go to the nearest router with
the content. In the worst case you go to the producer
  \end{itemize}
\end{itemize}

The problem with cache is the privacy. In fact, there is the possibility of
attacks:
\begin{itemize}
  \item Timing attacks
  \item Cache harvesting attacks
\end{itemize}

It's important to identify the adversary. An adversary could be:
\begin{itemize}
  \item Another host or router
  \item A malicious application on victim's device (like Worms, Trojans)
\end{itemize}

Also, and adversary could be near the consumer or near the producer in the
network. An adversary could identify the source of the requests. An example is
for example with governments or with IoT.

\paragraph{Scenario 1: Victim=Consumer}
For example, if you are in the same location, and you see a request to a
website, to confirm your doubts you can make the same request\footnote{also
know as \textit{interest}} and see the time of the reply. If the reply time is
sort it means that the content was recently cached in a router.

\paragraph{Scenario 2: Victim=Producer}
Now the Adversary is in an different position of the consumer, in a nearest one.
In this case if he try to time of the request the attack won't work properly,
but it can knowledge that somebody recently make a request because it's in the
cache, but it can't make this the second time, because the content will be in
the cache. It only knows that someone recently make the requests.

\paragraph{Scenario 3 Victims=Both}
In this case the attacker there is another actor, and the talkers are Alice and
Bob. Bob can asks for the Alice public key, and adversaries can listen for the
keys because they're fetched in the routers. And adversary can require the
Alice keys to know if Bob has requested recently the Alice key.

This if it's not in a internal network cloud be an attack.

\subsection{Countermeasures}

To solve this problem could be delete the cache content, but it's a bad idea,
because the network will return to be content centric. The strength of CCN is
the distributed content service.

There are three solutions to this problem:
\begin{enumerate}
  \item A solution could be caching but with delayed requests. But how long
should we display the requests? We could delay the request as long as the
first time that   took to fetch the first time the content. In this way it's
impossible distinguish a content that \textit{hop} to the producer and the
content that it's in the router cache. This will still improve the bandwidth
utilization, but it won't change the latency.

  \item ??

  \item Another solution to this problem could come from the \textit{producer}:
we can make the content of the producer unpredictable, so the attacker cannot
guess the origin of the content.

\end{enumerate}

So there are two types of traffic: private and non-private traffic. Also there
are different communication types:
\begin{itemize}
  \item Low-latency (interactive) traffic. We could use unpredictable content
name, but what if the are many people that want the same streaming? Like a
subscribed TV.
  \item Content distribution traffic; details in paper, IEEE ICDCS'13. We can
introduce different solutions:
  \begin{itemize}
    \item Random delay. You deliver the content with some random delay. It's an
interesting approach, because if the adversary ask for the first time the
content he will wait for the reply delay, otherwise he will have different
delays everytime. The problem it's that it's not the best solution for this
problem.
    \item Content-specific delay.
  \end{itemize}

  \item The user can ask to the producer to send private content.
\end{itemize}

There is another exploitation. You can use the pit to use the collision to
understand the requests that a user make.

\section{Dos/DDoS in CCN}

What is the Dos/DDos resistance of CCN?
Some current DoS + DDoS attacks become irrelevant in the CCN architecture.
Reasons:
\begin{itemize}
  \item Content caching mitigates targeted DoS. The networks absorbs the
attacks. A problem is when the attacker ask for different content, but in this
case it will be clear that it's an attacker.
  \item Content is not forwarded without prior PIT state set up by interest(s)
  \item Multiple interests for the same content are collapsed
  \item Only one copy of content per "interested" interface is returned
  \item Consumer can't be "hosed" with unsolicited content
\end{itemize}

There are also attack on the infrastructure
\begin{itemize}
  \item Loop-holing/black-holing
  \item Interest flooding
  \item Router resource exhaustion
\end{itemize}

Attack on Consumers plus router caches:
\begin{itemize}
  \item Content flooding
  \item Cache pollution
  \item Content/cache poisoning. This could be a problem because an attacker
can ask for random content, and this will use router and resources. What can
we do?
\begin{itemize}
  \item Unilateral rate limiting/throttling. The producer will take only
interests coming from areas without a lot of invalid requests.
  \item Collaborative rate limiting/throttling. In this case router share
information about prefixes and rates information. The problem it's that this
collaboration it's not free because you need to share this information with the
other router, sending other data.
\end{itemize}
There are different problem for content poisoning.
\begin{itemize}
  \item Adversary is on the path to producer. An evil router can change the
content in the cache and deliver to many user. This problem it's called
\textit{amplification}.
  \begin{itemize}
    \item Intercepts genuine interest, replies with fake content
    \item Content settles in routers
  \end{itemize}
  \item Adversary is not on the path to producer
  \begin{itemize}
    \item Anticipates demand for content
    \item Issues own interest(s), replies with fake content
    \item Content settles in routers
  \end{itemize}
\end{itemize}
\end{itemize}
