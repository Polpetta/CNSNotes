\section{Feature Space Construction for Malware Analysis}
Malware has some mechanism to copy themselves, and usually they use some system
bug to steal data.

Malware different from the different use of system call or the different type
of classes or object used.
In Android, general attacks are:
\begin{itemize}
  \item Premium SMS/Call: incur financial loss
  \item Theft of device details
  \item Identity theft
  \item Payment details: financial loss to user
  \item Credential theft
  \item Stealing personal data
\end{itemize}

A problem with Android APK it's the ability from the attacker to re-sign the
package with modified data.
So repackaging programs with malware it's something common.

\subsection{Shannon Entropy}
\textbf{Shannon Entropy}: A measure of average information content.

Properties:
\begin{itemize}
  \item Entropy depends on frequency of occurrence and not on order of
occurrence
\item AABBBBCD, ABCDABBB, ABBBACD have the same entropy
\item Permutation of an input shalll not change its entropy
\item Similarly, AABBBBCD and ABCCCCDD shall have same entropy
\item Malware variants created by transposition of instructions can have same
entropy values
\item Replacing and instruction by another shall not affect program entropy as
long as both instruction are likely to be used with almost same probability
\item Obfuscation of a program may affect entropy
\end{itemize}

System like gmail or other online service doesn't check the attachements of a
program because if it's a malware it can affect the whole system. System like
gmail checks the "magic bit" of the file, but the attacker can change the magic
bit of the program to fool the system.

Malicious file have a different type of entropy in confront of a legitimate
file. Also malware have statistically improbabile features in the code.
So a solution could be calculate the entropy of every feature, link the entropy
with a ranking system.
